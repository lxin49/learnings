% 导入字体、预设环境等
\input "/home/lxin49/Documents/tex/include/chnfonts.tex"
\input "/home/lxin49/Documents/tex/include/env.tex"

% 设置页码位置
\setuppagenumbering[location={bottom,middle}]

\starttext

\title{学习torch过程中的一些记录}

\section{用法}

\subsection{@}
\startitemize
\item output = tensor1 @ tensor2
\item output = tensor1.matmul(tensor2)
\item output = torch.matmul(tensor1,tensor2)
\item output = torch.matmul(tensor1,tensor2,out=output)
\stopitemize

\subsection{tensorboard}
from torch.utils.tensorboard import SummayWriter

\subsection{一种np的升维操作}
masks = mask == obj_ids[:, None, None]
mask是一个二维数组
obj_ids是一个一维数组,里面的值均在mask的值范围内
得到的masks是一个三维数组

\section{函数}

\startitemize
\item torch.rand_like
\item torch.randn
\item torch.randint
\item torch.mul
\item torch.add_
\item torch.from_numpy
\item torch.scatter_
\item torch.nn.Linear, torch.nn.Sigmoid, ...

\stopitemize

\subsection{torch.scatter_}
官方文档给出了3维张量操作说明:

self[index[i][j][k]][j][k] = src[i][j][k]  , if dim == 0

self[i][index[i][j][k]][k] = src[i][j][k]  , if dim == 1

self[i][j][index[i][j][k]] = src[i][j][k]  , if dim == 2

\stoptext