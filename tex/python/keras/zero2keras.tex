% 导入基础文件,包括字体、预定义环境,
% 页面大小文件暂时不用,故暂不导入
\input "/home/lxin49/Documents/tex/include/chnfonts.tex"
\input "/home/lxin49/Documents/tex/include/env.tex"
%\input "/home/lxin49/Documents/tex/include/pagesl.tex"

% 特殊要求修改
\setuppagenumbering[location={bottom,middle}]

% 开始正文
\starttext

\title{从零开始学习Keras}

\section{为什么要写这份文档}

写这份文档的首要目的是积累技术栈,从无到有,从零到一。对于机器学习,传统的ML方法,有一些认识,在这里不展开。DL的方法,一开始了解一些基本概念,什么是神经元、感知机、卷积、池化,但还是比较片面,想要深入地了解和学习DL相关知识,于是开始Keras的学习。

写这份文档的时候,已经把Keras囫囵吞枣地看了一遍,主要材料是keras.io的官方教程,除了API的文档,把整个流程做了一次大致的了解。整个完成后的感觉是,了解怎么使用Keras仅仅是第一步,能了解一些API的调用;而运用一些偏底层的API能稍微增加一些关于原理性的理解,以及增加使用API的熟练度,偶发性地增加些许成就感。除此以外,关于理论方面的,比如梯度下降、卷积、池化等还是局限于已获取的知识点,没有增加。又比如评估函数(metrics)、损失函数(losses)、优化器(optimizors)等等,了解得就更少,暂时只能使用而不知原理;各种同类函数之间的区别以及适用场景也不甚了解;在使用过程中有没有trick,也没有经验。这一块后期需要逐个了解评估函数、损失函数、优化器的优劣、适用场景等等,适用场景的目标是熟练掌握。

关于这份文档我的想法是分成三个部分:(1)怎么样直接调用API搭建网络,大概会包含网络搭建、数据灌入、训练及保存(这部分可能会以callbacks展开),尽量附带去讨论一些高级API不能实现的情形;(2)怎么样使用偏底层的API搭建网络,大概会包括网络搭建、数据预处理,尽量附带去讨论关于梯度的一些话题;(3)可能会讨论一下超参,主要是keras-tuner的使用,如果能写得出来,会尽量写keras-tuner在自定义模型中的应用(大概率写不出来),另外作为一个附带部分,回顾一下在整个过程中遇到的tensorflow的偏底层的函数或接口。

最后我想通过这份文档,加深巩固记忆。当然这份文档中的知识点还是比较浅层的,后续可能会更新v2版本,以期待有更深入的理解。

{\bi 注1:由于本人tex掌握不熟练,本文档中的代码实现将采用截图方式}

{\bi 注2:所有代码中import的部分,首次导入会写,其余都假设已经导入}

{\bi 注3:文档中keras和tensorflow的版本均为2.6.0}

\section{怎么样通过高级API实现模型}

这部分主要想的是这么几个内容:(1)通过高级API实现网络搭建,尽可能把API提供的搭建方式都进行罗列,可能会简单讨论高级API不能实现的几种情况(我尽量吧);(2)网络搭建好后怎么把数据灌入,这里不是讨论预处理,而是讨论大量数据下,不能将数据一次性读入到内存,通过使用队列或者生成器的方式把数据灌入模型;(3)模型训练、保存,将以callbacks展开

\subsection{通过高级API实现网络搭建}

\subsubsection{简单的网络搭建}

通过Keras高级API实现网络搭建有两种方法,也可以说有三种方法,要么是借助keras.Sequential实现,要么是借助FunctionalAPI实现,其中有两种都是通过keras.Sequential实现,所以也可以说是两种方法。下面是代码:

\placefigure[force,nonumber]{\bi 创建网络}\externalfigure[fig/1_create_net.png][width=325pt]

其中model1, model2, model3均为3层全链接层的结构,那这3个模型有没有区别呢?答案是肯定的。明显的区别是model2比model1多了一个layers.InputLayer,而model3则是keras.Input,layers.InputLayer和keras.Input的作用是一样的。那不明显的区别有没有呢?请尝试运行下面的代码:

\placefigure[force,nonumber]{\bi 查看网络结构}\externalfigure[fig/2_show_architecture.png][width=185pt]

通过运行可以发现,尝试查看model1的网络结构时报错了,model2的网络结构比model3少了一个InputLayer。因为model1没有经过build或fit方法,网络结构虽然成型,但由于没有输入层的形状没有确定,权重参数无法初始化,故无法输出网络结构及参数统计。model2可能是调用Sequential的缘故,所有Sequential模型在显示模型结构时都不显示输入层的大小。

那如果想看model1的网络结构,除了用model2的方法外,还有其他方法吗?有的。

\placefigure[force,nonumber]{\bi 构建模型}\externalfigure[fig/3_build_model.png][width=195pt]

需要注意的是,.build传入的是(batch,input_shape)

\subsubsection{网络搭建的几个小例子}

全链接层(Dense)的例子在简单的网络搭建中已举例,这个部分再找一些其他例子,目前在CV领域使用得比较多的基础模块是卷积和池化,在NLP领域使用得比较多的基础模块是RNN,LSTM,GRU,以及反卷积。反卷积在CV领域的图像对抗或者生成网络中应用得也比较广泛。在下面的例子中,仅仅是列举,具体的用法及参数不在此处罗列。

{\bf 卷积+池化+全链接}

\placefigure[force,nonumber]{}\externalfigure[fig/4_CNN.png][width=555pt]

{\bf 卷积+全局池化+全链接}

\placefigure[force,nonumber]{}\externalfigure[fig/5_CNN.png][width=560pt]

{\bf SimpleRNN/LSTM/GRU}

\placefigure[force,nonumber]{}\externalfigure[fig/6_RNN.png][width=350pt]

{\bf 反卷积}

\placefigure[force,nonumber]{}\externalfigure[fig/7_transpose2d.png][width=390pt]

上面这几个小例子是比较常见的,也是复杂模型搭建的基础,需要反复熟悉。此外,模型搭建,特别是成熟模型,需要不停尝试去搭建,才能有一些感觉。

\subsubsection{实现微型的残差网络}

\placefigure[force,nonumber]{}\externalfigure[fig/8_resnet.png][width=305pt]

\subsubsection{实现微型的对抗生成网络}

\placefigure[force,nonumber]{\bi 鉴别网络}\externalfigure[fig/9_discriminator.png][width=450pt]

\placefigure[force,nonumber]{\bi 生成网络}\externalfigure[fig/10_generator.png][width=495pt]

\placefigure[force,nonumber]{\bi 对抗生成网络}\externalfigure[fig/11_GAN.png][width=595pt]

在写这个微模型的时候,不停犯错,犯错的地方主要包括self.__init__()中的inputs和outputs,compile()里的super().compile(),训练时输入的形状没做reshape,训练时输入的dtype没有强制转为float32。除了犯错,我还有一个比较大的疑惑,那就是生成网络中fakes的标签应该以0还是1传入训练?其实我在mnist上都试过,效果不行。

\subsubsection{小结}

这个部分主要回顾了:
\startitemize
{\bi
\item 两个创建模型的方法:keras.Sequential和keras.Model,需要注意输入层的选择
\item 常见层使用和简易网络搭建:Dense,Conv2D,MaxPool2D,SimpleRNN,LSTM,GRU等层,全连接+卷积池化搭建
\item 不常见层的介绍和稍复杂的网络搭建:Conv2DTranspose层,残差网络,生成对抗网络
}
\stopitemize

这个部分的最后一个示例代码,里面涉及到两块内容:一是通过继承keras.Model自定义模型,里面使用的主要是一些偏底层的API。二是在这个GAN模型里自定义了compile函数,用于接收优化器参数;train_step函数,用于定义训练过程,包括损失计算、反向传播(评估函数虽然在这个例子里没有体现,但也是在train_step里通过metrics_fn进行计算,并在return里返回),与model.fit对应;call函数,与model(x)对应。

为什么会用对抗生成网络的例子,因为这个例子只能自定义,通过keras.Sequential或者keras.Model创建的网络只能接收一个优化器,而对抗生成网络需要多个优化器甚至损失函数,凡是需要多个优化器或者损失函数的,需要继承后自定义创建。

当然,DL和Keras里还有很多东西,特别是在NLP领域里,这个部分几乎没怎么提到。先在CV领域,有一些了解,再逐渐熟悉NLP领域。

\subsection{数据处理方式}

这个部分,我原不想讨论预处理,只想讨论数据灌入,少量数据通过numpy数组,中大量数据通过API或者自定义squeue实现。但由于后两种方式可能涉及到数据对齐的问题,便会在中间穿插预处理的方法。结构上会分成三部分,依次是numpy数组,Image,Text。

\subsubsection{NumPy数组处理}

在tensorflow或者keras中处理NumPy数组,主要有两个问题,灌入模型和预处理。先看灌入模型:

\placefigure[force,nonumber]{\bi NumPy数组批量导入}\externalfigure[fig/12_tensor_from_slices.png][width=415pt]

接下来看预处理:

\placefigure[force,nonumber]{\bi 归一化/标准化}\externalfigure[fig/13_normalization.png][width=550pt]

csv文件的灌入:

\placefigure[force,nonumber]{\bi 从csv批量读取数据}\externalfigure[fig/14_make_csv_dataset.png][width=500pt]

\subsubsection{图像数据处理}

图像数据处理和NumPy数据处理类似,直接上操作。灌入模型:

\placefigure[force,nonumber]{\bi 从文件夹读取图像数据}\externalfigure[fig/15_image_dataset.png][width=470pt]

图像数据的预处理主要包括归一化和标准化,与NumPy数组操作类似。另外有的操作是图像增强,包括:旋转,剪裁等等。

\placefigure[force,nonumber]{\bi 图像增强}\externalfigure[fig/16_images_preprocessing.png][width=435pt]

\subsubsection{文本数据处理}

文本数据处理和以上两类数据处理有所不同,主要在于文本编码。照惯例,先上灌入模型:

\placefigure[force,nonumber]{\bi 从文件夹读取文本数据}\externalfigure[fig/17_text_dataset.png][width=505pt]

由于中文与英文等的断字断句位置情况不同,一般直接读取txt文件无法有效断字、断句并形成词组或词向量。按我的习惯和喜好,更愿意采用队列方式去读取文件。

\placefigure[force,nonumber]{\bi 手动创建队列并实现预处理}\externalfigure[fig/18_text_sequence.png][width=505pt]

\subsubsection{小结}

这个部分主要是接触了3种类型的数据:NumPy数据,图像数据,文本数据。本质上来说,图像数据也是一种NumPy数据,固定形式为(高,宽,通道)的三维NumPy数据。文本数据一般无法直接使用,需要转换为词向量(本质也是一种NumPy数据),我已知的有jieba先分词,再用one_hot转为向量,也有用gensim模块进行word2vec训练的(这部分我还没接触过,以后再接触),也有用tensorflow或者pytorch的Embedding层进行训练词向量的(在接触文本预处理的部分在知乎看了点,不多)。

下面小结:

\startitemize
{\bi
\item NumPy数据的特征缩放(归一化/标准化)
\item 图像数据的旋转,缩放,裁剪
\item 文本数据的编码
\item 三种文件格式的读取
\item 队列对象的构建,继承自keras.utils.Sequence,至少两个方法__len__, __getitem__
}
\stopitemize

\subsection{模型的训练及保存}

在完成了模型搭建和数据准备之后,就可以开始模型训练、模型训练过程的可视化查看,模型训练完成后的模型保存。

\subsubsection{模型训练}

\placefigure[force,nonumber]{\bi 模型训练}{\externalfigure[fig/19_model_fit.png][width=480pt]}

模型训练中比较重要的部分是model.compile,这个函数里可以一般有3个参数,optimizer(优化器),loss(损失函数),metrics(评估函数),这3个参数里优化器和损失函数是必要的,评估函数可以没有。如果是自定义模型或者是自定义损失函数(评估函数),可以不存在损失函数和评估函数,甚至是优化器。(可能在第三部分会涉及)

\subsubsection{模型训练可视化 - callbacks}

\placefigure[force,nonumber]{\bi callbacks}{\externalfigure[fig/20_callbacks.png][width=585pt]}

\subsubsection{模型保存和载入}

\placefigure[force,nonumber]{\bi 保存和载入}{\externalfigure[fig/21_save_load_model.png][width=445pt]}

\subsubsection{小结}

\startitemize
{\bi
\item model.fit
\item callbacks
\item model.save, model.save_weights, keras.models.load_model, model.load_weights
}
\stopitemize

\section{怎么样通过偏底层API实现模型}

这份文档的主要内容集中在第二部分。这部分的原意是只想讨论模型搭建中偏底层API的使用,这个内容其实在第二部分的GAN网络介绍时中有体现,这部分会更详细地回顾一下。额外的就是回顾下损失函数和评估函数的偏底层API使用。

\subsection{模型搭建的偏底层API使用}

除了GAN网络,再给一个VGG的例子。这个VGG里面主要包括__init__,compile,call,train_step这四个函数,在这个VGG类里面,compile和train_step不是必须的,因为都没涉及到自定义的部分,如果有则需要添加对应内容。还有一个test_step函数未在代码中体现,对应的功能是model.fit中的validation_data部分,如果验证集的运行与训练集运行有区别,可以通过自定义test_step方式实现。此外,该模型中包含了一个名为create_blocks的函数,用于封装部分重复操作。

除了上面的函数及封装,VGG里面包含了一些预处理的功能,包括随机裁剪、中心裁剪、特征缩放几个功能,如有其他预处理需求,在相应模块增加。

\placefigure[force,nonumber]{\bi 手写模型}{\externalfigure[fig/22_vgg.png][width=535pt]}

\subsection{损失函数自定义的偏底层API使用}

\placefigure[force,nonumber]{\bi 自定义损失函数}{\externalfigure[fig/23_custom_loss.png][width=440pt]}

确实我对自定义Loss不熟悉,只能把自定义损失函数的类先放在这里。

我又去看了下文档,比较重要的应该是call函数。call函数的运行就决定了损失值大小,如果有y_true和y_pred以外的参数,可以在__init__函数里进行传参。

\subsection{评估函数自定义的偏底层API使用}

\placefigure[force,nonumber]{\bi 自定义评估函数}{\externalfigure[fig/24_custom_metric.png][width=550pt]}

确实我对自定义Metric不熟悉,只能把自定义评估函数的类先放在这里。

\subsection{小结}

这部分主要回顾了自定义Model,自定义Loss,自定义Metric,除了Model我写了一个例子,另外两个都只列示了写法,并没有写例子,是因为确实没有什么比人家框架更好的算法。另外,Model的保存因为少写了get_config和from_config而折腾了半天,图我也不打算补了,先就这样。需要注意,from_config是类方法,另外用callbacks.ModelCheckpoint保存的是model_weights。

\section{关于超参}

好吧,最后我部分我水了,其实是没有学透彻。全篇完。

\stoptext
